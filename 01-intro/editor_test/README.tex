% Author: Seongjin Lee 
% Hanyang University, Seoul, Korea 
% esos.hanyang.ac.kr 
% 2016-09-01
% note: some slides are adopted from  \url{www.cs.stevens.edu/~jschauma/631A/}
% https://github.com/resourceful/lecture_sysprog/

\documentclass{article}
\usepackage{kotex}
\usepackage{hyperref}
\hypersetup{pdfauthor={Seongjin Lee (insight@gnu.ac.kr)},
            pdfsubject={Editor Skill Test},
            pdfkeywords={vi, emacs, editor, skill test},
            pdfcreator={Seongjin Lee},
            hidelinks}




\title{Vi and Emacs Skill Test}
\author{Seongjin Lee}
\date{\today} 

\begin{document}
\maketitle 

\noindent
\fbox{
\parbox{\textwidth}{vi 또는 emacs를 이용하여 “blkparse.txt”를 연 뒤, 다음 지시사항들을 따라 하시기 바랍 니다. 동일한 경로에 “blktrace.txt”도 함께 위치시키시기 바랍니다.}
}

\begin{enumerate}
\item 51번째 line으로 커서를 이동하시오.
\item ``blktrace'' 문자열을 아래 방향으로 한 번 검색하여 커서를 이동하시오.
\item 현재커서위치에서한line을복사한뒤, 복사한문자열을현재커서위치의다음 line으로 삽입하시오.
\item 현재 커서 위치에서 위 방향으로 ``blkparse'' 문자열을 검색하여 5번 커서를 이동하시 오. 도착한 커서 위치의 line 한 줄을 지우시오.
\item 24~29 line 6줄을 복사한 뒤, 31번 line 시작 위치에 붙여 넣으시오.
\item 38번째 line으로 커서를 이동한 뒤, 일곱 줄을 지우시오.
\item 해당 파일의 첫 행으로 커서를 옮긴 뒤, 현재 커서 위치부터 나오는 ``blkparse'' 문자 열 10개를 ``blocktrace\_parser''로 치환하시오.
\item ``blktrace.txt'' 파일의 본문을, 현재 파일의 444번 Line 시작 위치에 삽입하시오.
\item 해당 파일의 첫 행으로 커서를 옮긴 뒤, 다음과 같이 매크로를 작성하고 수행하시오.
\begin{enumerate}
\item btrace 문자열 검색 $\rightarrow$ line 끝으로 이동$\rightarrow$ btrace is here 입력
\item 작성한 매크로를 7번 반복하시오.
\end{enumerate}
\item 현재까지 변경된 문서를 ``이름\_본인학번.txt''로 저장한 뒤, \href{mailto:insight@gnu.ac.kr}{insight@gnu.ac.kr}로 보내시요. (Ex. 이순신\_2011234567.txt)
\end{enumerate}
\end{document}