% Author: Seongjin Lee 
% Hanyang University, Seoul, Korea 
% esos.hanyang.ac.kr 

\documentclass[12pt]{article}
\usepackage{kotex}



\title{Editor Evaluation Excerscise}
\author{Seongjin Lee}
\date{2016-12-14} 

\begin{document}
\maketitle

vi 또는 emacs를 이용하여 ``blkparse.txt''를 연 뒤, 다음 지시사항을 따라 하시기 바랍니다. 동일한 경로에 ``blktrace.txt''도 있는지 확인합니다.

\begin{enumerate}
\item 51번째 줄로 이동
\item ``blktrace'' 문자열을 아래 방향으로 한 번 검색하여 커서 이동
\item 현재커서위치에서한line을복사한뒤,복사한문자열을현재커서위치의다음 line으로 삽입하시오.
\item  현재 커서 위치에서 위 방향으로 ``blkparse'' 문자열을 검색하여 5번 커서를 이동하시 오. 도착한 커서 위치의 line 한 줄을 지우시오.
\item 24~29 line 6줄을 복사한 뒤, 31번 line 시작 위치에 붙여 넣으시오.
\item  38번째 line으로 커서를 이동한 뒤, 일곱 줄을 지우시오.
\item  해당 파일의 첫 행으로 커서를 옮긴 뒤, 현재 커서 위치부터 나오는 ``blkparse'' 문자 열 10개를 ``blocktrace\_parser''로 치환하시오.
\item  ``blktrace.txt'' 파일의 본문을, 현재 파일의 444번 Line 시작 위치에 삽입하시오.
\item  해당 파일의 첫 행으로 커서를 옮긴 뒤, 다음과 같이 매크로를 작성하고 수행하시오.
  \begin{enumerate}
  \item btrace 문자열 검색 $\rightarrow$ line 끝으로 이동 $\rightarrow$ btrace is here 입력
  \item 작성한 매크로를 7번 반복하시오.
  \end{enumerate}
\item 새로운 이름으로 저장
\end{enumerate}


\end{document}
